\documentclass[12pt,a4paper]{article}

\usepackage{amsmath, amsthm, amssymb, amsbsy, amsfonts}
\usepackage[plainpages=false,pdfpagelabels]{hyperref}
\hypersetup{
    colorlinks,
    citecolor=black,
    filecolor=black,
    linkcolor=black,
    urlcolor=blue
}
\usepackage{verbatim}
\usepackage{enumitem}
\usepackage[utf8x]{inputenc}

% Redefine the \vec command to use bold font instead of an arrow
\renewcommand\vec[1]{\mathbf{#1}}

%\def\nl{\hfill\break\null}
\oddsidemargin=0.3cm
\topmargin=-1cm
\textwidth=15cm
\textheight=23cm
\parindent=0cm
\parskip=1mm


\usepackage{graphicx}
\DeclareMathOperator*{\argmin}{argmin}

\newenvironment{enumialpha}{\begin{enumerate}
  \def\theenumi{\alph{enumi}}
  \def\labelenumi{(\theenumi)}}{\end{enumerate}}

\newenvironment{enumiiroman}{\begin{enumerate}
  \def\theenumii{\roman{enumii}}
  \def\labelenumii{(\theenumii)}}{\end{enumerate}}


\begin{document}

\begin{tabular*}{15cm}{@{}l@{\extracolsep{\fill}}r}
  Albert-Ludwigs-Universit\"at Freiburg, Institut f\"ur Informatik \\
PD Dr. Cyrill Stachniss \\  Lecture: Robot Mapping \\
  Winter term 2013 
\end{tabular*}

\bigskip


\begin{center}
{\bf \Large Sheet 2}

{\large Topic: Bayes Filter}

Submission deadline: Nov. 4, 2013\\
Submit to: \texttt{robotmappingtutors@informatik.uni-freiburg.de}
\end{center}

\subsubsection*{Exercise: Bayes Filter}

A robot is equipped with a manipulator to paint an object.  Furthermore,
the robot has a sensor to detect whether the object is colored or blank.
Neither the manipulation unit nor the sensor are perfect.

From previous experience you know that the robot succeeds in painting a
blank object with a probability of
\begin{equation*}
p( x_{t+1} = \text{colored} \mid x_t = \text{blank}, u_{t+1} = \text{paint} ) = 0.9,
\end{equation*}
where $x_{t+1}$ is the state of the object after executing a painting action,
$u_{t+1}$ is the control command, and $x_t$ is the state of the object
before performing the action.

The probability that the sensor indicates that the object is colored
although it is blank is given by $p(z=\text{colored} \mid
x=\text{blank}) = 0.2$, and the probability that the sensor correctly
detects a colored object is given by $p(z=\text{colored} \mid x =
\text{colored}) = 0.7$.

Unfortunately, you have no knowledge about the current state of the
object. However, after the robot performed a painting action the sensor
of the robot indicates that the object is colored.

Compute the probability that the object is still blank after the robot
has performed an action to paint it. Use an appropriate prior
distribution and justify your choice.

\end{document}
