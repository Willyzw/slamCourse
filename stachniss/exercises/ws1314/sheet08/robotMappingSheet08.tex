\documentclass[12pt]{article}

\usepackage{amsmath, amsthm, amssymb, amsbsy, amsfonts}
\usepackage[plainpages=false,pdfpagelabels]{hyperref}
\hypersetup{
    colorlinks,
    citecolor=black,
    filecolor=black,
    linkcolor=black,
    urlcolor=blue
}
\usepackage{verbatim}
\usepackage{enumitem}
\usepackage[utf8x]{inputenc}

% Redefine the \vec command to use bold font instead of an arrow
\renewcommand\vec[1]{\mathbf{#1}}

%\def\nl{\hfill\break\null}
\oddsidemargin=0.3cm
\topmargin=-1cm
\textwidth=15cm
\textheight=23cm
\parindent=0cm
\parskip=1mm


\usepackage{graphicx}
\DeclareMathOperator*{\argmin}{argmin}

\newenvironment{enumialpha}{\begin{enumerate}
  \def\theenumi{\alph{enumi}}
  \def\labelenumi{(\theenumi)}}{\end{enumerate}}

\newenvironment{enumiiroman}{\begin{enumerate}
  \def\theenumii{\roman{enumii}}
  \def\labelenumii{(\theenumii)}}{\end{enumerate}}


\begin{document}

\begin{tabular*}{15cm}{@{}l@{\extracolsep{\fill}}r}
  Albert-Ludwigs-Universit\"at Freiburg, Institut f\"ur Informatik \\
PD Dr. Cyrill Stachniss \\  Lecture: Robot Mapping \\
  Winter term 2013 
\end{tabular*}

\bigskip


\begin{center}
{\bf \Large Sheet 8}

{\large Topic: FastSLAM}

Submission deadline: January, 7\\
Submit to: \texttt{robotmappingtutors@informatik.uni-freiburg.de}
\end{center}

\subsubsection*{Exercise: FastSLAM Implementation}

Implement the basic FastSLAM 1.0 algorithm as presented in the lecture.
Assume known correspondences and use a simple feature storage approach,
i.e., {\bf not} the tree data structure that yields an ${\cal O}(N \log
M)$ time complexity.

To support this task, we provide a small \emph{Octave} framework (see
course website).  The framework contains the following folders:

\begin{description}
\item [data]
  contains files representing the world definition and sensor readings.
\item [octave]
  contains the FastSLAM framework with stubs  to complete.
\item [plots]
  this folder is used to store images.
\end{description}

The below mentioned task should be implemented inside the framework in
the directory \texttt{octave} by completing the stubs.

After implementing the missing parts, you can run the FastSLAM system.
To do that, change into the directory octave and launch \emph{Octave}.
Type \texttt{fastslam} to start the main loop (this may take some time).
The script will produce plots of the state of the FastSLAM algorithm and
save them in the \texttt{plots} directory.  You can use the images for
debugging and to generate an animation. For example, you can use ffmpeg
from inside the plots directory as follows:
\begin{verbatim}
ffmpeg -r 10 -b 500000 -i fastslam_%03d.png fastslam.mp4
\end{verbatim}

Implement the correction step in \texttt{correction\_step.m}.  For the
noise in the sensor model, assume that $Q_t$ is a diagonal $2 \times 2$
matrix as follows
\begin{align*}
  Q_t &= \begin{pmatrix} 0.1 & 0 \\ 0 & 0.1 \end{pmatrix}.
\end{align*}

Some implementation tips:
\begin{itemize}
  \item
    Turn off the visualization to speed up the computation by
    commenting out the line \texttt{plot\_state(...)} in the file
    \texttt{fastslam.m}.
  \item
    While debugging, run the filter only for a few steps by
    replacing the for-loop in \texttt{fastslam.m} by
    something along the lines of \texttt{for t = 1:50}.
  \item
    When converting implementations containing for-loops into a
    vectorized form it often helps to draw the dimensions of the data
    involved on a sheet of paper.
  \item
    Many of the functions in \emph{Octave} can handle matrices and
    compute values along the rows or columns of a matrix. Some useful
    functions that support this are \texttt{sum}, \texttt{cumsum},
    \texttt{sqrt}, \texttt{sin}, \texttt{cos}, and many others.
\end{itemize}

\end{document}
