\documentclass[12pt,a4paper]{article}

\usepackage{amsmath, amsthm, amssymb, amsbsy, amsfonts}
\usepackage[plainpages=false,pdfpagelabels]{hyperref}
\hypersetup{
    colorlinks,
    citecolor=black,
    filecolor=black,
    linkcolor=black,
    urlcolor=blue
}
\usepackage{verbatim}
\usepackage{enumitem}
\usepackage[utf8x]{inputenc}

% Redefine the \vec command to use bold font instead of an arrow
\renewcommand\vec[1]{\mathbf{#1}}

%\def\nl{\hfill\break\null}
\oddsidemargin=0.3cm
\topmargin=-1cm
\textwidth=15cm
\textheight=23cm
\parindent=0cm
\parskip=1mm


\usepackage{graphicx}
\DeclareMathOperator*{\argmin}{argmin}

\newenvironment{enumialpha}{\begin{enumerate}
  \def\theenumi{\alph{enumi}}
  \def\labelenumi{(\theenumi)}}{\end{enumerate}}

\newenvironment{enumiiroman}{\begin{enumerate}
  \def\theenumii{\roman{enumii}}
  \def\labelenumii{(\theenumii)}}{\end{enumerate}}


\begin{document}

\begin{tabular*}{15cm}{@{}l@{\extracolsep{\fill}}r}
  Albert-Ludwigs-Universit\"at Freiburg, Institut f\"ur Informatik \\
PD Dr. Cyrill Stachniss \\  Lecture: Robot Mapping \\
  Winter term 2013 
\end{tabular*}

\bigskip


\begin{center}
{\bf \Large Sheet 3}

{\large Topic: Extended Kalman Filter SLAM}

Submission deadline: Nov.~11\\
Submit to: \texttt{robotmappingtutors@informatik.uni-freiburg.de}
\end{center}

\subsubsection*{Exercise 1: Bayes Filter and EKF}

\begin{enumialpha}
\item Describe briefly the two main steps of the Bayes filter in your own words.

\item Describe briefly the meaning of the following probability density functions: $p(x_t \mid u_t, x_{t-1})$, $p(z_t \mid x_t)$, and $\text{bel}(x_t)$, which are processed by the Bayes filter. 

\item Specify the (normal) distributions that correspond to the above mentioned three terms in EKF SLAM.

\item Explain in a few sentences all of the components of the EKF SLAM algorithm, \mbox{i.\ e.}, $\mu_t$, $\Sigma_t$, $g$, $G^x_t$, $G_t$, $R^x_t$, $R_t$, $h$, $H_t$, $Q_t$, $K_t$ and why they are needed. Specify the dimensionality of these components.
\end{enumialpha}

\subsubsection*{Exercise 2: Jacobians}

\begin{enumialpha}
    \item Derive the Jacobian matrix~$G^x_t$ of the noise-free motion function~$g$ with respect to the pose of the robot. Use the odometry motion model as in exercise sheet~1:

\begin{equation*}
     \left(\begin{array}{c} x_t \\ y_t \\ \theta_t \end{array}\right) = \left(\begin{array}{c} x_{t-1} \\ y_{t-1} \\ \theta_{t-1} \end{array}\right) + \left(\begin{array}{c} \delta_{trans} \cos(\theta_{t-1} + \delta_{rot1}) \\ \delta_{trans} \sin(\theta_{t-1} + \delta_{rot1}) \\ \delta_{rot1} + \delta_{rot2} \end{array}\right).
\end{equation*}

Do not use Octave for this part of the exercise.

    \item Derive the Jacobian matrix~$^{\textrm{low}}H^i_t$ of the noise-free sensor function~$h$ corresponding to the $i^{\textrm{th}}$ measurement:

\begin{equation*}
 h(\bar{\mu}_t,j) =  z^i_t = \left(\begin{array}{c} r^i_t \\ \phi^i_t  \end{array}\right) = \left(\begin{array}{c} \sqrt{(\bar{\mu}_{j,x} - \bar{\mu}_{t,x})^2 + (\bar{\mu}_{j,y} - \bar{\mu}_{t,y})^2} \\ \textrm{atan2}(\bar{\mu}_{j,y} - \bar{\mu}_{t,y}, \bar{\mu}_{j,x} - \bar{\mu}_{t,x})-\bar{\mu}_{t,\theta} \end{array}\right),
\end{equation*}
where $(\bar{\mu}_{j,x}, \bar{\mu}_{j,y})^T$ is the pose of the $j^{\textrm{th}}$ landmark, $(\bar{\mu}_{t,x}, \bar{\mu}_{t,y}, \bar{\mu}_{t,\theta})^T$ is the pose of the robot at time $t$, and $r^i_t$ and $\phi^i_t$ are respectively the observed range and bearing of the landmark. Do not use Octave for this part of the exercise.\\

\emph{Hint:} use $\frac{\partial}{\partial x}\textrm{atan2}(y,x) = \frac{-y}{x^2+y^2}$, and $\frac{\partial}{\partial y}\textrm{atan2}(y,x) = \frac{x}{x^2+y^2}$.

\begin{comment}
\begin{equation*}
\begin{array}{l}
\displaystyle\frac{\partial}{\partial x}\textrm{atan2}(y,x) = \frac{-y}{x^2+y^2},\\ \\
\displaystyle\frac{\partial}{\partial y}\textrm{atan2}(y,x) = \frac{x}{x^2+y^2}.\\
\end{array}
\end{equation*}
\end{comment}

\end{enumialpha}

\end{document}
