\documentclass[12pt]{article}

\usepackage{amsmath, amsthm, amssymb, amsbsy, amsfonts}
\usepackage[plainpages=false,pdfpagelabels]{hyperref}
\hypersetup{
    colorlinks,
    citecolor=black,
    filecolor=black,
    linkcolor=black,
    urlcolor=blue
}
\usepackage{verbatim}
\usepackage{enumitem}
\usepackage[utf8x]{inputenc}

% Redefine the \vec command to use bold font instead of an arrow
\renewcommand\vec[1]{\mathbf{#1}}

%\def\nl{\hfill\break\null}
\oddsidemargin=0.3cm
\topmargin=-1cm
\textwidth=15cm
\textheight=23cm
\parindent=0cm
\parskip=1mm


\usepackage{graphicx}
\DeclareMathOperator*{\argmin}{argmin}

\newenvironment{enumialpha}{\begin{enumerate}
  \def\theenumi{\alph{enumi}}
  \def\labelenumi{(\theenumi)}}{\end{enumerate}}

\newenvironment{enumiiroman}{\begin{enumerate}
  \def\theenumii{\roman{enumii}}
  \def\labelenumii{(\theenumii)}}{\end{enumerate}}


\begin{document}

\begin{tabular*}{15cm}{@{}l@{\extracolsep{\fill}}r}
  Albert-Ludwigs-Universit\"at Freiburg, Institut f\"ur Informatik \\
PD Dr. Cyrill Stachniss \\  Lecture: Robot Mapping \\
  Winter term 2013 
\end{tabular*}

\bigskip


\begin{center}
{\bf \Large Sheet 5}

{\large Topic: The Unscented Transform}

Submission deadline: Nov.~25\\
Submit to: \texttt{robotmappingtutors@informatik.uni-freiburg.de}
\end{center}

\subsubsection*{Exercise: The Unscented Transform}

Implement the Unscented Transform (using \emph{Octave}). The
implementation should consist of two parts, computing the sigma
points and recovering the transformed Gaussian:

\begin{enumialpha}

\item Implement the function in \texttt{compute\_sigma\_points.m},
  which samples the $2n+1$ sigma points given the mean vector and
  covariance matrix. You should also compute the corresponding point
  weights $w_m^{\lbrack i \rbrack}$ and $w_c^{\lbrack i \rbrack}$ for
  $i= 0, \dots, 2n$.
    
\item Implement the function in \texttt{recover\_gaussian.m} to
  compute the mean and covariance of the resulting distribution given
  the transformed sigma points and their weights.
\end{enumialpha}

To support this task, we provide a small \emph{Octave} framework (see
course website).  The above-mentioned tasks should be implemented
inside the framework in the directory \texttt{octave} by completing
the stubs. After implementing the missing parts, you can test your
solution by running the main script. The program will produce a plot
containing both the original and transformed distributions and save it
in the \texttt{plots} directory.

The code provides three different functions describing transformations
applied to the distribution. Test your implementation on each of them
by uncommenting the corresponding parts in \texttt{transform.m}.

After completing the exercise, try other transformations by implementing
them in \texttt{transform.m}. Moreover, you can change the parameters 
($\alpha$ and $\kappa$) in \texttt{main.m} for computing $\lambda$ and 
inspect how this affects the sampled sigma points.

Hint: To compute the square root of the covariance matrix in
\emph{Octave}, you can use the function \texttt{sqrtm}. Alternatively,
you can compute the Cholesky decomposition using \texttt{chol}.

\end{document}
