\documentclass[12pt]{article}

\usepackage{amsmath, amsthm, amssymb, amsbsy, amsfonts}
\usepackage[plainpages=false,pdfpagelabels]{hyperref}
\hypersetup{
    colorlinks,
    citecolor=black,
    filecolor=black,
    linkcolor=black,
    urlcolor=blue
}
\usepackage{verbatim}
\usepackage{enumitem}
\usepackage[utf8x]{inputenc}

% Redefine the \vec command to use bold font instead of an arrow
\renewcommand\vec[1]{\mathbf{#1}}

%\def\nl{\hfill\break\null}
\oddsidemargin=0.3cm
\topmargin=-1cm
\textwidth=15cm
\textheight=23cm
\parindent=0cm
\parskip=1mm


\usepackage{graphicx}
\DeclareMathOperator*{\argmin}{argmin}

\newenvironment{enumialpha}{\begin{enumerate}
  \def\theenumi{\alph{enumi}}
  \def\labelenumi{(\theenumi)}}{\end{enumerate}}

\newenvironment{enumiiroman}{\begin{enumerate}
  \def\theenumii{\roman{enumii}}
  \def\labelenumii{(\theenumii)}}{\end{enumerate}}

%\def\nl{\hfill\break\null}
\oddsidemargin=0.3cm
\topmargin=-1cm
\textwidth=15cm
\textheight=23cm
\parindent=0cm
\parskip=1mm


\usepackage{graphicx}
\DeclareMathOperator*{\argmin}{argmin}

\newenvironment{enumialpha}{\begin{enumerate}
  \def\theenumi{\alph{enumi}}
  \def\labelenumi{(\theenumi)}}{\end{enumerate}}

\newenvironment{enumiiroman}{\begin{enumerate}
  \def\theenumii{\roman{enumii}}
  \def\labelenumii{(\theenumii)}}{\end{enumerate}}


\begin{document}

\begin{tabular*}{15cm}{l@{\extracolsep{\fill}}r}
  Albert-Ludwigs-Universit\"at Freiburg, Institut f\"ur Informatik \\
PD Dr. Cyrill Stachniss \\  Lecture: Robot Mapping \\
  Winter term 2012 
\end{tabular*}



\bigskip


\begin{center}
{\bf \Large Sheet 5}

{\large Topic: Particle Filter}

Submission deadline: December, 3\\
Submit to: \texttt{robotmappingtutors@informatik.uni-freiburg.de}
\end{center}

\subsubsection*{Exercise 1: Particle Filter}

\begin{enumialpha}
\item Describe briefly the main differences between the particle filter
  and the Extended Kalman filter for state estimation.
\item
  Discuss briefly the advantages of the low variance re-sampling strategy.
\end{enumialpha}

\subsubsection*{Exercise 2: Particle Filter Implementation}

First, implement the prediction step of a particle by sampling the
motion of a robot given the distribution $p(x_t\mid u_{t-1},x_{t-1})$
and second, implement the re-sampling step.

\begin{enumialpha}
\item
  Implement the function in \texttt{prediction\_step.m}, which samples a
  motion for each particle according to the motion model and the given
  noise parameters.
\item
  Implement the function in \texttt{resample.m}, which re-samples the
  set of particles utilizing the low variance re-sampling method.
\end{enumialpha}

To support this task, we provide a small \emph{Octave} framework (see
course website). The above-mentioned tasks should be implemented inside
the framework in the directory \texttt{octave} by completing the stubs.
After implementing the missing parts, you can test your solutions by
running the script in \texttt{motion.m} for the prediction step and
\texttt{resampling.m} for the re-sample step.  The script
\texttt{motion.m} will produce plots of the position of the particles
and save them in the \texttt{plots} directory.


Some implementation tips:
\begin{itemize}
  \item
    The function \texttt{normrnd($\mu$, $\sigma$)} allows to draw
    samples from a Gaussian with mean $\mu$ and standard deviation
    $\sigma$.
  \item The function \texttt{unifrnd($a$, $b$)} generates random samples
    from the uniform distribution on $\left[ a, b \right]$.
  %\item
    %When converting implementations containing for-loops into a
    %vectorized form it often helps to draw the dimensions of the data
    %involved on a sheet of paper.
  \item
    Many of the functions in \emph{Octave} can handle matrices and
    compute values along the rows or columns of a matrix. Some useful
    functions that support this are \texttt{sum}, \texttt{cumsum},
    \texttt{sqrt}, \texttt{sin}, \texttt{cos}, and many others.
\end{itemize}

\end{document}
