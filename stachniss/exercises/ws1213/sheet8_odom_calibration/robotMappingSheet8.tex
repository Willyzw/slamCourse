\documentclass[12pt]{article}

\usepackage{amsmath, amsthm, amssymb, amsbsy, amsfonts}
\usepackage[plainpages=false,pdfpagelabels]{hyperref}
\hypersetup{
    colorlinks,
    citecolor=black,
    filecolor=black,
    linkcolor=black,
    urlcolor=blue
}
\usepackage{verbatim}
\usepackage{enumitem}
\usepackage[utf8x]{inputenc}

% Redefine the \vec command to use bold font instead of an arrow
\renewcommand\vec[1]{\mathbf{#1}}

%\def\nl{\hfill\break\null}
\oddsidemargin=0.3cm
\topmargin=-1cm
\textwidth=15cm
\textheight=23cm
\parindent=0cm
\parskip=1mm


\usepackage{graphicx}
\DeclareMathOperator*{\argmin}{argmin}

\newenvironment{enumialpha}{\begin{enumerate}
  \def\theenumi{\alph{enumi}}
  \def\labelenumi{(\theenumi)}}{\end{enumerate}}

\newenvironment{enumiiroman}{\begin{enumerate}
  \def\theenumii{\roman{enumii}}
  \def\labelenumii{(\theenumii)}}{\end{enumerate}}

%\def\nl{\hfill\break\null}
\oddsidemargin=0.3cm
\topmargin=-1cm
\textwidth=15cm
\textheight=23cm
\parindent=0cm
\parskip=1mm


\usepackage{graphicx}
\DeclareMathOperator*{\argmin}{argmin}

\newenvironment{enumialpha}{\begin{enumerate}
  \def\theenumi{\alph{enumi}}
  \def\labelenumi{(\theenumi)}}{\end{enumerate}}

\newenvironment{enumiiroman}{\begin{enumerate}
  \def\theenumii{\roman{enumii}}
  \def\labelenumii{(\theenumii)}}{\end{enumerate}}


\begin{document}

\begin{tabular*}{15cm}{l@{\extracolsep{\fill}}r}
  Albert-Ludwigs-Universit\"at Freiburg, Institut f\"ur Informatik \\
PD Dr. Cyrill Stachniss \\  Lecture: Robot Mapping \\
  Winter term 2012 
\end{tabular*}



\bigskip


\begin{center}
{\bf \Large Sheet 8}

{\large Topic: Least-Squares}

Submission deadline: January, 14\\
Submit to: \texttt{robotmappingtutors@informatik.uni-freiburg.de}
\end{center}

\subsubsection*{Exercise: Odometry Calibration}

Implement an odometry calibration tool based on a least-squares method as presented in the
lecture. To support this task, we provide a small \emph{Octave} framework (see
course website).  The framework contains the following folders:

\begin{description}
\item [data]
  contains the recorded raw odometry and the motion estimated by a
  scan-matcher for each time step.
\item [octave]
  contains the Octave framework with stubs to complete.
\item [plots]
  this folder is used to store images.
\end{description}

The below mentioned tasks should be implemented inside the framework in
the directory \texttt{octave} by completing the stubs:
\begin{itemize}
  \item
    Implement the functions in \texttt{ls\_calibrate\_odometry.m} for
    constructing and solving the least-squares system.
  \item
    Implement the function in \texttt{apply\_odometry\_correction.m} for
    applying the calibration matrix to a set of odometry measurements.
  \item
    Implement the function in \texttt{compute\_trajectory.m} for
    chaining up the relative odometry measurements.
\end{itemize}

After implementing the missing parts, you can run the framework.  To do
that, change into the directory octave and launch \emph{Octave}. To
start the main loop, type \texttt{LSCalibrateOdometry}.  The script will
produce a plot showing the trajectory of the raw odometry measurements,
the estimate obtained by scan-matching, and the odometry after applying
the calibration.  This plot will be saved in the \texttt{plots}
directory. 

The file odometry-calibration-result.png depicts the result that you should obtain.

%You can use the images for
%debugging and to generate an animation. For example, you can use ffmpeg
%from inside the plots directory as follows:
%\begin{verbatim}
%ffmpeg -r 10 -b 500000 -i gridmap_%03d.png gridmap.mp4
%\end{verbatim}


%Some implementation tips:
%\begin{itemize}
  %\item Use an inverse sensor model corresponding to laser range finders (see lecture slides). The corresponding $p_{free}$ and $p_{occ}$ values are specified in the \texttt{gridmap.m} script. Use $p_{occ}$ to update the occupancy value of cells that laser beam endpoints hit and $p_{free}$ for all other cells along the beam. Use the function \texttt{robotlaser\_as\_cartesian.m} to compute the Cartesian coordinates of the endpoints of a laser scan. The provided \texttt{bresenham.m} function can be used for computing the cells that lie along a laser beam in map coordinates.
  %\item
%Compute all occupancy value updates in log odds (not probabilities) so they can be added directly to the map.
  %\item
 %Test your implementation with a grid size of 0.5m. Once you are satisfied with your results, you can run the algorithm with an increased resolution (e.g. 0.1m), as this will take considerably more time.
  %\item
    %While debugging, run the algorithm only for a few steps by
    %replacing the for-loop in \texttt{gridmap.m} by
    %something along the lines of \texttt{for t = 1:10}.
  %\item
    %Many of the functions in \emph{Octave} can handle matrices and
    %compute values along the rows or columns of a matrix. Some useful
    %functions that support this are \texttt{sum},  \texttt{log},
    %\texttt{sqrt}, \texttt{sin}, \texttt{cos}, and many others.
%\end{itemize}

% Turn off the visualization to speed up the computation by commenting out the line \texttt{plot\_state(...} in the file \texttt{fastslam.m}.

\end{document}
