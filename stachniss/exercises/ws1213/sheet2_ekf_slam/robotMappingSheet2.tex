\documentclass[12pt]{article}

\usepackage{amsmath, amsthm, amssymb, amsbsy, amsfonts}
\usepackage[plainpages=false,pdfpagelabels]{hyperref}
\hypersetup{
    colorlinks,
    citecolor=black,
    filecolor=black,
    linkcolor=black,
    urlcolor=blue
}
\usepackage{verbatim}
\usepackage{enumitem}
\usepackage[utf8x]{inputenc}

% Redefine the \vec command to use bold font instead of an arrow
\renewcommand\vec[1]{\mathbf{#1}}

%\def\nl{\hfill\break\null}
\oddsidemargin=0.3cm
\topmargin=-1cm
\textwidth=15cm
\textheight=23cm
\parindent=0cm
\parskip=1mm


\usepackage{graphicx}
\DeclareMathOperator*{\argmin}{argmin}

\newenvironment{enumialpha}{\begin{enumerate}
  \def\theenumi{\alph{enumi}}
  \def\labelenumi{(\theenumi)}}{\end{enumerate}}

\newenvironment{enumiiroman}{\begin{enumerate}
  \def\theenumii{\roman{enumii}}
  \def\labelenumii{(\theenumii)}}{\end{enumerate}}

%\def\nl{\hfill\break\null}
\oddsidemargin=0.3cm
\topmargin=-1cm
\textwidth=15cm
\textheight=23cm
\parindent=0cm
\parskip=1mm


\usepackage{graphicx}
\DeclareMathOperator*{\argmin}{argmin}

\newenvironment{enumialpha}{\begin{enumerate}
  \def\theenumi{\alph{enumi}}
  \def\labelenumi{(\theenumi)}}{\end{enumerate}}

\newenvironment{enumiiroman}{\begin{enumerate}
  \def\theenumii{\roman{enumii}}
  \def\labelenumii{(\theenumii)}}{\end{enumerate}}


\begin{document}

\begin{tabular*}{15cm}{l@{\extracolsep{\fill}}r}
  Albert-Ludwigs-Universit\"at Freiburg, Institut f\"ur Informatik \\
PD Dr. Cyrill Stachniss \\  Lecture: Robot Mapping \\
  Winter term 2012 
\end{tabular*}



\bigskip


\begin{center}
{\bf \Large Sheet 2}

{\large Topic: Extended Kalman Filter SLAM}

Submission deadline: Nov. 12$^\textrm{th}$ for Exercises 1 and 2, Nov. 19$^\textrm{th}$ for Exercise 3\\
Submit to: \texttt{robotmappingtutors@informatik.uni-freiburg.de}
\end{center}

%\subsubsection*{General Notice}

%- Q_t von u abhaengig machen als make em think aufgabe
%- G aufschreiben als pen n paper aufgabe und dann als folgeaufg das ergebnis implementieren
%- verteilungen im bayes filter mit entspr normalverteilungen in bezug bringen und dann eben die matrizen die da auftauchen erklaeren
%- matrizen des ekf erklaeren lassen

\subsubsection*{Exercise 1: Bayes Filter and EKF}

\begin{enumialpha}
\item Describe briefly the two main steps of the Bayes filter in your own words.

\item Describe briefly the meaning of the following probability density functions: $p(x_t \mid u_t, x_{t-1})$, $p(z_t \mid x_t)$, and $\text{bel}(x_t)$, which are processed by the Bayes filter. 

\item Specify the (normal) distributions that correspond to the above mentioned three terms in EKF SLAM.

%The Bayes filter processes three probability density functions, \mbox{i.\ e.}, \linebreak $p(x_t \mid u_t, x_{t-1})$, $p(z_t \mid x_t)$, and $\text{bel}(x_t)$. State the normal distributions of the EKF which correspond to these probabilities.

\item Explain in a few sentences all of the components of the EKF SLAM algorithm, \mbox{i.\ e.}, $\mu_t$, $\Sigma_t$, $g$, $G^x_t$, $G_t$, $R^x_t$, $R_t$, $h$, $H_t$, $Q_t$, $K_t$ and why they are needed. Specify the dimensionality of these components.
\end{enumialpha}

\subsubsection*{Exercise 2: Jacobians}

\begin{enumialpha}
    \item Derive the Jacobian matrix~$G^x_t$ of the noise-free motion function~$g$ with respect to the pose of the robot. Use the odometry motion model as in exercise sheet~1:

\begin{equation*}
     \left(\begin{array}{c} x_t \\ y_t \\ \theta_t \end{array}\right) = \left(\begin{array}{c} x_{t-1} \\ y_{t-1} \\ \theta_{t-1} \end{array}\right) + \left(\begin{array}{c} \delta_{trans} \cos(\theta_{t-1} + \delta_{rot1}) \\ \delta_{trans} \sin(\theta_{t-1} + \delta_{rot1}) \\ \delta_{rot1} + \delta_{rot2} \end{array}\right).
\end{equation*}

Do not use Octave for this part of the exercise.

    \item Derive the Jacobian matrix~$^{\textrm{low}}H^i_t$ of the noise-free sensor function~$h$ corresponding to the $i^{\textrm{th}}$ measurement:

\begin{equation*}
 h(\bar{\mu}_t,j) =  z^i_t = \left(\begin{array}{c} r^i_t \\ \phi^i_t  \end{array}\right) = \left(\begin{array}{c} \sqrt{(\bar{\mu}_{j,x} - \bar{\mu}_{t,x})^2 + (\bar{\mu}_{j,y} - \bar{\mu}_{t,y})^2} \\ \textrm{atan2}(\bar{\mu}_{j,y} - \bar{\mu}_{t,y}, \bar{\mu}_{j,x} - \bar{\mu}_{t,x})-\bar{\mu}_{t,\theta} \end{array}\right),
\end{equation*}
where $(\bar{\mu}_{j,x}, \bar{\mu}_{j,y})^T$ is the pose of the $j^{\textrm{th}}$ landmark, $(\bar{\mu}_{t,x}, \bar{\mu}_{t,y}, \bar{\mu}_{t,\theta})^T$ is the pose of the robot at time $t$, and $r^i_t$ and $\phi^i_t$ are respectively the observed range and bearing of the landmark. Do not use Octave for this part of the exercise.\\

\emph{Hint:} use $\frac{\partial}{\partial x}\textrm{atan2}(y,x) = \frac{-y}{x^2+y^2}$, and $\frac{\partial}{\partial y}\textrm{atan2}(y,x) = \frac{x}{x^2+y^2}$.

\begin{comment}
\begin{equation*}
\begin{array}{l}
\displaystyle\frac{\partial}{\partial x}\textrm{atan2}(y,x) = \frac{-y}{x^2+y^2},\\ \\
\displaystyle\frac{\partial}{\partial y}\textrm{atan2}(y,x) = \frac{x}{x^2+y^2}.\\
\end{array}
\end{equation*}
\end{comment}

\end{enumialpha}

\subsubsection*{Exercise 3: Implement an EKF SLAM System}

Implement an extended Kalman filter SLAM~(EKF SLAM) system.  To support
this task, we provide a small Octave framework (see course website).
The framework contains the following folders:

\begin{description}
\item [data]  contains files representing the world definition and
    sensor readings.
\item [octave]  contains the EKF SLAM framework with
    stubs  to complete.
\item [plots] this folder is used to store images.
\end{description}

The below mentioned tasks should be implemented inside the framework
in the directory \texttt{octave} by completing the stubs.

After implementing the missing parts, you can run the EKF SLAM
system. To do that, change into the directory \texttt{octave} and
launch \emph{Octave}. Type \texttt{ekf\_slam} to start the main loop
(this may take some time). The program plots the current belief of the
robot (pose and landmarks) in the directory \texttt{plots}. You can
use the images for debugging and to generate an animation. For
example, you can use \emph{ffmpeg} from inside the \texttt{plots}
directory as follows:
\begin{verbatim}
ffmpeg -r 10 -b 500000 -i ekf_%03d.png ekf_slam.mp4
\end{verbatim}

\begin{enumialpha}

    \item Implement the prediction step of the EKF SLAM algorithm in the file\\
\texttt{prediction\_step.m}. Use the Jacobian~$G^x_t$ you derived above to construct the full Jacobian matrix~$G_t$. For the noise in the motion model, assume
\begin{equation*}
R^x_t = \left(\begin{array}{ccc} 0.1 & 0 & 0 \\ 0 & 0.1 & 0 \\ 0 & 0 & 0.01 \end{array}\right).
\end{equation*}

    \item Implement the correction step in the file \texttt{correction\_step.m}. The argument $z$ of this function is a struct array containing $m$ landmark observations made at time step $t$. Each observation $z(i)$ has an id $z(i)$.\emph{id}, a range $z(i)$.\emph{range}, and a bearing $z(i)$.\emph{bearing}.

Iterate over all measurements ($i=1,\dots, m$) and compute $H^i_t$ using the Jacobian you derived above. You should compute a block Jacobian matrix $H_t$ by stacking the $H^i_t$ matrices corresponding to the individual measurements. Use it to compute the Kalman gain and update the system mean and covariance \emph{after} the for-loop. For the noise in the sensor model, assume that $Q_t$ is a diagonal square matrix as follows
\begin{equation*}
Q_t = \left(\begin{array}{cccc} 0.01 & 0 & 0 & \ldots \\ 0 & 0.01 & 0 & \ldots \\ 0 & 0 & 0.01 & \ldots \\ \vdots & \vdots & \vdots & \ddots  \end{array}\right) \in \mathbb{R}^{2m \times 2m}.
\end{equation*}
\end{enumialpha}

%(of size $2N+3 \times 2m$)

Some implementation tips:
\begin{itemize}
    \item Turn off the visualization to speed up the computation by
        commenting out the line \texttt{plot\_state(...} in the file
        \texttt{ekf\_slam.m}.
    \item While debugging, run the filter only for a few steps by
        replacing the for-loop in \texttt{ekf\_slam.m} by
        something along the lines of \texttt{for t = 1:50}.
    \item The command \texttt{repmat} allows you to replicate a given
        matrix in many different ways and is magnitudes faster than
        using for-loops.
    \item When converting implementations containing for-loops into a
        vectorized form it often helps to draw the dimensions of the
        data involved on a sheet of paper.
    \item Many of the functions in \emph{Octave} can handle matrices and
        compute values along the rows or columns of a matrix. Some
        useful functions that support this are \texttt{sum},
        \texttt{sqrt}, and many others.
\end{itemize}



\end{document}

