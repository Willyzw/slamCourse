\documentclass[12pt]{article}

\usepackage{amsmath, amsthm, amssymb, amsbsy, amsfonts}
\usepackage[plainpages=false,pdfpagelabels]{hyperref}
\hypersetup{
    colorlinks,
    citecolor=black,
    filecolor=black,
    linkcolor=black,
    urlcolor=blue
}
\usepackage{verbatim}
\usepackage{enumitem}
\usepackage[utf8x]{inputenc}

% Redefine the \vec command to use bold font instead of an arrow
\renewcommand\vec[1]{\mathbf{#1}}

%\def\nl{\hfill\break\null}
\oddsidemargin=0.3cm
\topmargin=-1cm
\textwidth=15cm
\textheight=23cm
\parindent=0cm
\parskip=1mm


\usepackage{graphicx}
\DeclareMathOperator*{\argmin}{argmin}

\newenvironment{enumialpha}{\begin{enumerate}
  \def\theenumi{\alph{enumi}}
  \def\labelenumi{(\theenumi)}}{\end{enumerate}}

\newenvironment{enumiiroman}{\begin{enumerate}
  \def\theenumii{\roman{enumii}}
  \def\labelenumii{(\theenumii)}}{\end{enumerate}}

%\def\nl{\hfill\break\null}
\oddsidemargin=0.3cm
\topmargin=-1cm
\textwidth=15cm
\textheight=23cm
\parindent=0cm
\parskip=1mm


\usepackage{graphicx}
\DeclareMathOperator*{\argmin}{argmin}

\newenvironment{enumialpha}{\begin{enumerate}
  \def\theenumi{\alph{enumi}}
  \def\labelenumi{(\theenumi)}}{\end{enumerate}}

\newenvironment{enumiiroman}{\begin{enumerate}
  \def\theenumii{\roman{enumii}}
  \def\labelenumii{(\theenumii)}}{\end{enumerate}}


\begin{document}

\begin{tabular*}{15cm}{l@{\extracolsep{\fill}}r}
  Albert-Ludwigs-Universit\"at Freiburg, Institut f\"ur Informatik \\
PD Dr. Cyrill Stachniss \\  Lecture: Robot Mapping \\
  Winter term 2012 
\end{tabular*}



\bigskip


\begin{center}
{\bf \Large Sheet 9}

{\large Topic: Graph-Based SLAM}

Submission deadline: February, 4\\
Submit to: \texttt{robotmappingtutors@informatik.uni-freiburg.de}
\end{center}

\subsubsection*{Exercise: Graph-Based SLAM}

Implement a least-squares method to address SLAM in its graph-based
formulation. To support this task, we provide a small \emph{Octave} framework (see
course website).  The framework contains the following folders:

\begin{description}
\item [data]
  contains several datasets, each gives the measurements of one SLAM
  problem
\item [octave]
  contains the Octave framework with stubs to complete.
\item [plots]
  this folder is used to store images.
\end{description}

The below mentioned tasks should be implemented inside the framework in
the directory \texttt{octave} by completing the stubs:
\begin{itemize}
\item
  Implement the function in \texttt{compute\_global\_error.m} for computing the
  current error.
\item
  Implement the function in \texttt{linearize\_pose\_pose\_constraint.m} for
  computing the error and the Jacobian of a pose-pose constraint.
\item
  Implement the function in
  \texttt{linearize\_pose\_landmark\_constraint.m} for computing the
  error and the Jacobian of a pose-landmark constraint.
\item
  Implement the function in \texttt{linearize\_and\_solve.m} for
  constructing and solving the linear approximation.
\item
  Implement the update of the state vector and the stopping criterion in
  \texttt{lsSLAM.m}.  A possible choice for the stopping criterion is
  $\| \Delta x \|_\infty < \epsilon$, i.e., the maximum absolute $\Delta
  x$ does not exceed $\epsilon$.
\end{itemize}

After implementing the missing parts, you can run the framework.  To do
that, change into the directory octave and launch \emph{Octave}. To
start the main loop, type \texttt{lsSLAM}. The script will produce a
plot showing the positions of the robot and (if available) the positions
of the landmarks in each iteration.  These plots will be saved in the
\texttt{plots} directory. 

\clearpage
The file $\langle$name of the dataset$\rangle$.png depicts the result
that you should obtain after convergence for each dataset.
Additionally, the initial and the final error for each dataset should be
approximately:\\[1ex]
\begin{tabular}{|l|c|c|}
  \hline
  dataset & initial error & final error\\\hline\hline
  simulation-pose-pose.dat & 138862234 & 8269\\\hline
  simulation-pose-landmark.dat & 3030 & 474\\\hline
  intel.dat & 1795139 & 360\\\hline
  dlr.dat & 369655336 & 56860\\\hline
\end{tabular}\\

The state vector contains the following entities:
\begin{itemize}
  \item pose of the robot: $\mathbf{x}_i = (x_i \; y_i \; \theta_i)^T$\\
    Hint: You may use the function $\mathrm{v2t}(\cdot)$ and
    $\mathrm{t2v}(\cdot)$:
    \begin{align*}
    \mathrm{v2t}(\mathbf{x}_i) &= 
    \begin{pmatrix}
      R_i &\mathbf{t}_i\\
      \mathbf{0} & 1
    \end{pmatrix}
    =
    \begin{pmatrix}
      \cos(\theta_i) & -\sin(\theta_i) & x_i\\
      \sin(\theta_i) & \cos(\theta_i) & y_i\\
      0 & 0 & 1
    \end{pmatrix}
    = X_i\\
    \mathrm{t2v}(X_i) &= \mathbf{x}_i
    \end{align*}
  \item position of a landmark: $\mathbf{x}_l = (x_l \; y_l)^T$
\end{itemize}

We consider the following error functions:
\begin{itemize}
  \item pose-pose constraint:
    $\mathbf{e}_\mathit{ij} = \mathrm{t2v}(Z^{-1}_\mathit{ij} (X_i^{-1}
    X_j))$, where $Z_\mathit{ij} = \mathrm{v2t}(\mathbf{z}_\mathit{ij})$
    is the transformation matrix of the measurement $\mathbf{z}_{ij}^T =
    (\mathbf{t}_{ij}^T, \theta_{ij})$.\\
    Hint: For computing the Jacobian, write the error function with
    rotation matrices and translation vectors:
    \begin{align*}
      \mathbf{e}_\mathit{ij} &=
      \begin{pmatrix}
        R_{ij}^T (R_i^T(\mathbf{t}_j-\mathbf{t}_i)-\mathbf{t}_{ij})\\
        \theta_j -\theta_i - \theta_{ij}\\
      \end{pmatrix}
    \end{align*}
  \item pose-landmark constraint:
    $\mathbf{e}_\mathit{il} = R_i^T (\mathbf{x}_l - \mathbf{t}_i) -
    \mathbf{z}_{il}$
\end{itemize}

Some implementation tips:
\begin{itemize}
\item
  The functions v2t($\cdot$) and t2v($\cdot$) to convert
  between a pose vector and a transformation matrix are available in the
  Octave framework.
\item
  You may first implement the functions in
  \texttt{linearize\_pose\_pose\_constraint.m} and apply the framework
  to the datasets which do not contain landmarks.
\item
  When solving the linear system exploit the sparseness of the matrix.
  You may resort to the backslash operator.
\item
  Many of the functions in \emph{Octave} can handle matrices and
  compute values along the rows or columns of a matrix. Some useful
  functions that support this are \texttt{max}, \texttt{abs}, \texttt{sum},  \texttt{log},
  \texttt{sqrt}, \texttt{sin}, \texttt{cos}, and many others.
\end{itemize}

\end{document}
