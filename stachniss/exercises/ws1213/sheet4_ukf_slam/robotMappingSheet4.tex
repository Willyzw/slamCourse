\documentclass[12pt]{article}

\usepackage{amsmath, amsthm, amssymb, amsbsy, amsfonts}
\usepackage[plainpages=false,pdfpagelabels]{hyperref}
\hypersetup{
    colorlinks,
    citecolor=black,
    filecolor=black,
    linkcolor=black,
    urlcolor=blue
}
\usepackage{verbatim}
\usepackage{enumitem}
\usepackage[utf8x]{inputenc}

% Redefine the \vec command to use bold font instead of an arrow
\renewcommand\vec[1]{\mathbf{#1}}

%\def\nl{\hfill\break\null}
\oddsidemargin=0.3cm
\topmargin=-1cm
\textwidth=15cm
\textheight=23cm
\parindent=0cm
\parskip=1mm


\usepackage{graphicx}
\DeclareMathOperator*{\argmin}{argmin}

\newenvironment{enumialpha}{\begin{enumerate}
  \def\theenumi{\alph{enumi}}
  \def\labelenumi{(\theenumi)}}{\end{enumerate}}

\newenvironment{enumiiroman}{\begin{enumerate}
  \def\theenumii{\roman{enumii}}
  \def\labelenumii{(\theenumii)}}{\end{enumerate}}

%\def\nl{\hfill\break\null}
\oddsidemargin=0.3cm
\topmargin=-1cm
\textwidth=15cm
\textheight=23cm
\parindent=0cm
\parskip=1mm


\usepackage{graphicx}
\DeclareMathOperator*{\argmin}{argmin}

\newenvironment{enumialpha}{\begin{enumerate}
  \def\theenumi{\alph{enumi}}
  \def\labelenumi{(\theenumi)}}{\end{enumerate}}

\newenvironment{enumiiroman}{\begin{enumerate}
  \def\theenumii{\roman{enumii}}
  \def\labelenumii{(\theenumii)}}{\end{enumerate}}


\begin{document}

\begin{tabular*}{15cm}{l@{\extracolsep{\fill}}r}
  Albert-Ludwigs-Universit\"at Freiburg, Institut f\"ur Informatik \\
PD Dr. Cyrill Stachniss \\  Lecture: Robot Mapping \\
  Winter term 2012 
\end{tabular*}



\bigskip


\begin{center}
{\bf \Large Sheet 4}

{\large Topic: Unscented Kalman Fitler SLAM}

Submission deadline: December, 3\\
Submit to: \texttt{robotmappingtutors@informatik.uni-freiburg.de}
\end{center}

\subsubsection*{Exercise: UKF SLAM}

Implement an unscented Kalman filter SLAM~(UKF SLAM) system. You should complete the following three parts:

\begin{enumialpha}

\item Implement the function in \texttt{compute\_sigma\_points.m},
  which samples sigma points given a mean vector and
  covariance matrix according to the unscented transform.

\item Implement the prediction step of the filter by completing the function in \texttt{prediction\_step.m} to
  compute the mean and covariance after incorporating the odometry motion command.

\item Implement the correction step in \texttt{correction\_step.m} to
  update the belief after each sensor measurement according to the UKF SLAM algorithm.

\end{enumialpha}

To support this task, we provide a small \emph{Octave} framework (see
course website). The above-mentioned tasks should be implemented
inside the framework in the directory \texttt{octave} by completing
the stubs. After implementing the missing parts, you can test your
solution by running the script in \texttt{ukf\_slam.m}. The program will produce plots of the robot pose and map estimates and save them
in the \texttt{plots} directory.

Note that, as opposed to the EKF SLAM system you implemented in sheet 2, here the state vector and covariance matrix are incrementally grown with each newly-observed landmark. The mean estimates of the landmark poses are stacked in $\mu_t$ in the order by which they were observed by the robot, as described in the \emph{map} vector in the framework.

Some implementation tips:
\begin{itemize}
  \item
    Be careful when averaging angles. One way to average a set of
      angles $\left\{ \theta_1, \cdots \theta_N \right\}$ given their weights $\left\{ w_1, \cdots w_N \right\}$ is to first compute the weighted sum of the unit-vectors of each rotation
      \begin{eqnarray*}
        \bar x = \sum_{i=i}^N w_i \cos(\theta_i),\\
        \bar y = \sum_{i=i}^N w_i \sin(\theta_i).
      \end{eqnarray*}
      Then the average angle $\bar \theta$ is given by
      \begin{equation*}
        \bar \theta = \mathrm{atan2}(\bar y, \bar x).
      \end{equation*}
    \item Use the following weights when recovering the mean and covariance from sampled sigma points:
      \begin{eqnarray*}
	w_m^{\lbrack0\rbrack} = w_c^{\lbrack0\rbrack} = \frac{\lambda}{n+\lambda},\\
	w_m^{\lbrack i \rbrack} = w_c^{\lbrack i \rbrack} = \frac{1}{2(n+\lambda)},
      \end{eqnarray*}
	where $i = 1, \dots, 2n+1$ and $n$ is the dimensionality of the system.
    \item Turn off the visualization to speed up the computation by
        commenting out the line \texttt{plot\_state(...} in the file
        \texttt{ukf\_slam.m}.
    \item While debugging, run the filter only for a few steps by
        replacing the for-loop in \texttt{ukf\_slam.m} by
        something along the lines of \texttt{for t = 1:50}.
    \item The command \texttt{repmat} allows you to replicate a given
        matrix in many different ways and is magnitudes faster than
        using for-loops.
    \item When converting implementations containing for-loops into a
        vectorized form it often helps to draw the dimensions of the
        data involved on a sheet of paper.
    \item Many of the functions in \emph{Octave} can handle matrices and
        compute values along the rows or columns of a matrix. Some
        useful functions that support this are \texttt{sum},
        \texttt{sqrt}, \texttt{sin}, \texttt{cos}, and many others.
\end{itemize}

\end{document}

