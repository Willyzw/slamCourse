\documentclass[12pt]{article}

\usepackage{amsmath, amsthm, amssymb, amsbsy, amsfonts}
\usepackage[plainpages=false,pdfpagelabels]{hyperref}
\hypersetup{
    colorlinks,
    citecolor=black,
    filecolor=black,
    linkcolor=black,
    urlcolor=blue
}
\usepackage{verbatim}
\usepackage{enumitem}
\usepackage[utf8x]{inputenc}

% Redefine the \vec command to use bold font instead of an arrow
\renewcommand\vec[1]{\mathbf{#1}}

%\def\nl{\hfill\break\null}
\oddsidemargin=0.3cm
\topmargin=-1cm
\textwidth=15cm
\textheight=23cm
\parindent=0cm
\parskip=1mm


\usepackage{graphicx}
\DeclareMathOperator*{\argmin}{argmin}

\newenvironment{enumialpha}{\begin{enumerate}
  \def\theenumi{\alph{enumi}}
  \def\labelenumi{(\theenumi)}}{\end{enumerate}}

\newenvironment{enumiiroman}{\begin{enumerate}
  \def\theenumii{\roman{enumii}}
  \def\labelenumii{(\theenumii)}}{\end{enumerate}}

%\def\nl{\hfill\break\null}
\oddsidemargin=0.3cm
\topmargin=-1cm
\textwidth=15cm
\textheight=23cm
\parindent=0cm
\parskip=1mm


\usepackage{graphicx}
\DeclareMathOperator*{\argmin}{argmin}

\newenvironment{enumialpha}{\begin{enumerate}
  \def\theenumi{\alph{enumi}}
  \def\labelenumi{(\theenumi)}}{\end{enumerate}}

\newenvironment{enumiiroman}{\begin{enumerate}
  \def\theenumii{\roman{enumii}}
  \def\labelenumii{(\theenumii)}}{\end{enumerate}}


\begin{document}

\begin{tabular*}{15cm}{l@{\extracolsep{\fill}}r}
  Albert-Ludwigs-Universit\"at Freiburg, Institut f\"ur Informatik \\
PD Dr. Cyrill Stachniss \\  Lecture: Robot Mapping \\
  Winter term 2012 
\end{tabular*}



\bigskip


\begin{center}
{\bf \Large Sheet 3}

{\large Topic: The Unscented Transform}

Submission deadline: November, 19\\
Submit to: \texttt{robotmappingtutors@informatik.uni-freiburg.de}
\end{center}

\subsubsection*{Exercise: The Unscented Transform}

Implement the Unscented Transform (using \emph{Octave}). The
implementation should consist of two parts, computing the sigma
points and recovering the transformed Gaussian:

\begin{enumialpha}

\item Implement the function in \texttt{compute\_sigma\_points.m},
  which samples the $2n+1$ sigma points given the mean vector and
  covariance matrix. You should also compute the corresponding point
  weights $w_m^{\lbrack i \rbrack}$ and $w_c^{\lbrack i \rbrack}$ for
  $i= 0, \dots, 2n$.
    
\item Implement the function in \texttt{recover\_gaussian.m} to
  compute the mean and covariance of the resulting distribution given
  the transformed sigma points and their weights.
\end{enumialpha}

To support this task, we provide a small \emph{Octave} framework (see
course website).  The above-mentioned tasks should be implemented
inside the framework in the directory \texttt{octave} by completing
the stubs. After implementing the missing parts, you can test your
solution by running the main script. The program will produce a plot
containing both the original and transformed distributions and save it
in the \texttt{plots} directory.

The code provides three different functions describing transformations
applied to the distribution. Test your implementation on each of them
by uncommenting the corresponding parts in \texttt{transform.m}.

After completing the exercise, try other transformations by implementing
them in \texttt{transform.m}. Moreover, you can change the parameters 
($\alpha$ and $\kappa$) in \texttt{main.m} for computing $\lambda$ and 
inspect how this affects the sampled sigma points.

Hint: to compute the square root of the covariance matrix in
\emph{Octave}, you can use the function \texttt{sqrtm}. Alternatively,
you can compute the Cholesky decomposition using \texttt{chol}.






\begin{comment}
Some implementation tips:
\begin{itemize}
  \item
    Be careful when averaging angles. One way to average a set of
      angles $\left\{ \theta_1, \cdots \theta_N \right\}$ works as
      follows. First we sum the unit-vectors of each rotation.
      \begin{eqnarray*}
        \bar x = \sum_{i=i}^N \cos(\theta_i)\\
        \bar y = \sum_{i=i}^N \sin(\theta_i)
      \end{eqnarray*}
      Then the average angle $\bar \theta$ can be determined by
      \begin{equation*}
        \bar \theta = \mathrm{atan2}(\bar y, \bar x).
      \end{equation*}
    \item Turn off the visualization to speed up the computation by
        commenting out the line \texttt{plot\_state(...} in the file
        \texttt{ukf\_slam.m}.
    \item While debugging, run the filter only for a few steps by
        replacing the for-loop in \texttt{ukf\_slam.m} by
        something along the lines of \texttt{for t = 1:50}.
    \item The command \texttt{repmat} allows you to replicate a given
        matrix in many different ways and is magnitudes faster than
        using for-loops.
    \item When converting implementations containing for-loops into a
        vectorized form it often helps to draw the dimensions of the
        data involved on a sheet of paper.
    \item Many of the functions in \emph{Octave} can handle matrices and
        compute values along the rows or columns of a matrix. Some
        useful functions that support this are \texttt{sum},
        \texttt{sqrt}, and many others.
\end{itemize}
\end{comment}

\end{document}

