\documentclass[12pt]{article}

\usepackage{amsmath, amsthm, amssymb, amsbsy, amsfonts}
\usepackage[plainpages=false,pdfpagelabels]{hyperref}
\hypersetup{
    colorlinks,
    citecolor=black,
    filecolor=black,
    linkcolor=black,
    urlcolor=blue
}
\usepackage{verbatim}
\usepackage{enumitem}
\usepackage[utf8x]{inputenc}

% Redefine the \vec command to use bold font instead of an arrow
\renewcommand\vec[1]{\mathbf{#1}}

%\def\nl{\hfill\break\null}
\oddsidemargin=0.3cm
\topmargin=-1cm
\textwidth=15cm
\textheight=23cm
\parindent=0cm
\parskip=1mm


\usepackage{graphicx}
\DeclareMathOperator*{\argmin}{argmin}

\newenvironment{enumialpha}{\begin{enumerate}
  \def\theenumi{\alph{enumi}}
  \def\labelenumi{(\theenumi)}}{\end{enumerate}}

\newenvironment{enumiiroman}{\begin{enumerate}
  \def\theenumii{\roman{enumii}}
  \def\labelenumii{(\theenumii)}}{\end{enumerate}}

%\def\nl{\hfill\break\null}
\oddsidemargin=0.3cm
\topmargin=-1cm
\textwidth=15cm
\textheight=23cm
\parindent=0cm
\parskip=1mm


\usepackage{graphicx}
\DeclareMathOperator*{\argmin}{argmin}

\newenvironment{enumialpha}{\begin{enumerate}
  \def\theenumi{\alph{enumi}}
  \def\labelenumi{(\theenumi)}}{\end{enumerate}}

\newenvironment{enumiiroman}{\begin{enumerate}
  \def\theenumii{\roman{enumii}}
  \def\labelenumii{(\theenumii)}}{\end{enumerate}}


\begin{document}

\begin{tabular*}{15cm}{l@{\extracolsep{\fill}}r}
  Albert-Ludwigs-Universit\"at Freiburg, Institut f\"ur Informatik \\
PD Dr. Cyrill Stachniss \\  Lecture: Robot Mapping \\
  Winter term 2012 
\end{tabular*}



\bigskip


\begin{center}
{\bf \Large Sheet 10}

{\large Topic: Graph-Based SLAM}

Submission deadline: February, 11\\
Submit to: \texttt{robotmappingtutors@informatik.uni-freiburg.de}
\end{center}

\subsubsection*{Exercise: Max-Mixture Least Squares SLAM}

Implement the max-mixture approximation for addressing multi-modal constraints in the context of least-squares, graph-based SLAM. To support this task, we provide a small \emph{Octave} framework (see
course website).  The framework contains the following folders:

\begin{description}
\item [data]
  contains several datasets, each gives the measurements of one SLAM
  problem
\item [octave]
  contains the Octave framework with stubs to complete.
\item [plots]
  this folder is used to store images.
\end{description}

The below mentioned tasks should be implemented inside the framework in
the directory \texttt{octave} by completing the stubs:
\begin{itemize}
\item
  Implement the function in \texttt{compute\_best\_mixture\_component.m} for selecting the most likely mode of a constraint based on the max-mixture formulation.
\item
  Implement the function in \texttt{compute\_global\_error.m} for computing the total squared error of a graph.
\item
  Implement the function in \texttt{linearize\_and\_solve.m} for constructing and solving the linear approximation.
\end{itemize}

After implementing the missing parts, you can run the framework.  To do that, change into the directory octave and launch \emph{Octave}. To start the main loop, type \texttt{maxmixlsSlam}. The script will produce a plot showing the positions of the robot in each iteration.  These plots will be saved in the \texttt{plots} directory. 

\clearpage
The file $\langle$name of the dataset$\rangle$.png depicts the result that you should obtain after convergence for each dataset. Additionally, the initial and the final error ($\sum\limits_i e_{i}^T\Omega_{i}e_{i}$) for each dataset should be
approximately:\\[1ex]
\begin{tabular}{|l|c|c|}
  \hline
  dataset & initial error & final error\\\hline\hline
  manhattan250.dat & 111& 6\\\hline
    manhattan500.dat& 322 & 16\\\hline
  manhattan1000a.dat & 3932& 32 \\\hline
  manhattan1000b.dat & 3932 & 32\\\hline
\end{tabular}\\


Use the following criterion when computing the most likely mixture component, $k^*$, of a constraint:
\begin{equation*}
k^* = \argmin_k (\frac{1}{2}e_{ij_k}^T\Omega_{ij_k}e_{ij_k} - \mathrm{log}(w_k) + \frac{1}{2}\mathrm{log}(|\Sigma_{ij_k}|))
\end{equation*}

Some implementation tips:
\begin{itemize}
\item
  You can use the \texttt{compute\_error\_pose\_pose\_constraint.m} function available to you to compute the error vector of a pose-pose constraint.
\item
  You can use the \texttt{linearize\_pose\_pose\_constraint.m} function available to you to compute the Jacobian of a pose-pose constraint.
\item
  Many of the functions in \emph{Octave} can handle matrices and
  compute values along the rows or columns of a matrix. Some useful
  functions that support this are \texttt{max}, \texttt{abs}, \texttt{sum},  \texttt{log},
  \texttt{sqrt}, \texttt{sin}, \texttt{cos}, and many others.
\end{itemize}

\end{document}
